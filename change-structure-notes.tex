\documentclass{article}

\usepackage[letterpaper,
  hratio=1:1,vratio=1:1,
  text={121mm,220mm},             % 5:9
  %text={121mm,212mm},             % 4:7
  %text={140mm,210mm},            % 2:3
  %text={152mm,228mm}, % 2:3
  %text={162mm,228mm}, % 5:7
]{geometry}

%% FONTS & TYPOGRAPHY

%% ---------- OPTION 1: CHARTER ----------
\usepackage[sups,scaled=.96]{XCharter}
\usepackage[scaled=1.010417965903645]{inconsolata}
\usepackage[scaled=0.95,proportional,semibold]{sourcesanspro}
%\usepackage[osf,scaled=1.007,scaled=1.033,scaled=1.02,]{AlegreyaSans}
\usepackage{eulervm}
\edef\zeu@Scale{0.99}
\PassOptionsToPackage{scaled=\zeu@Scale,bbscaled=\zeu@Scale,scrscaled=\zeu@Scale}{mathalfa}
\RequirePackage[cal=cm]{mathalfa}
%% Make sure \mathbold and \mathvar are defined.
\linespread{1.06667}            %eyeballed

%% %% ---------- OPTION 2: PALATINO ----------
%% \usepackage[scaled=.96,largesc,nohelv]{newpxtext}
%% \usepackage[scaled=1.0176]{biolinum}
%% \usepackage[scaled=0.964203422764601]{inconsolata}
%% \edef\zeu@Scale{.96}
%% %% HACK: newpxtext doesn't respect fontaxes' \tbfigures! :( but it _does_
%% %% define, eg., \tlfstyle! hm...
%% \usepackage{fontaxes}
%% \renewcommand\tbfigures{}
%% \renewcommand\lnfigures{}
%% %\linespread{1.023435141098595}
%% \linespread{1.08}              % eyeballed 2025-02-06
%% %% %% NEWPXMATH %%
%% %% \usepackage[scaled=.96,vvarbb,smallerops]{newpxmath}
%% %% \usepackage[bb=ams,bbscaled=.96,cal=cm]{mathalfa}
%% %% EULER MATH %%
%% \usepackage{eulervm}
%% \PassOptionsToPackage{scaled=\zeu@Scale,bbscaled=\zeu@Scale,scrscaled=\zeu@Scale}{mathalfa}
%% \usepackage[cal=cm]{mathalfa}

%% %% END FONTS & TYPOGRAPHY

\usepackage[dvipsnames]{xcolor}
%% cleveref must be loaded AFTER hyperref & amsmath, or it errors; and AFTER
%% amsthm and BEFORE we define theorem styles, otherwise (eg) it calls lemmas
%% "Theorem"s.
\usepackage{hyperref,url}
\usepackage{amsmath,amsthm}
\usepackage[nameinlink,noabbrev]{cleveref}
\usepackage{mathpartir}
\usepackage{mathtools}          %\Coloneqq, \dblcolon
\usepackage{amssymb}            %\multimap
%\usepackage{stmaryrd}           %\llbracket, \shortrightarrow
%\usepackage{array}              %\setlength\extrarowheight
%\usepackage{booktabs}           %\midrule

\theoremstyle{definition}
\newtheorem{definition}{Definition}
\theoremstyle{remark}
\newtheorem{remark}{Remark}

\newcommand\crefrangeconjunction{--} % use en-dashes for ranges.
\creflabelformat{equation}{#2#1#3}
\crefformat{footnote}{#2\footnotemark[#1]#3} % referencing footnotes

\newcommand\ensuretext[1]{{\ifmmode\text{#1}\else{#1}\fi}}
\newcommand\strong[1]{{\bfseries#1}}

\newcommand\defeq\triangleq
\newcommand\bnfeq{\Coloneqq}
\newcommand\bnfor{\mathrel{\,|\,}}

\newcommand\fnspace\:           %binder space
\newcommand\fn[1]{\lambda{#1}.\fnspace}
\newcommand\<\;                 %function application space

\newcommand\quantifierspace{\ }
\newcommand\quantify[1]{({#1})\quantifierspace}
\newcommand\fa[1]{\quantify{\forall #1}}
\newcommand\ex[1]{\quantify{\exists #1}}
\newcommand\faex[2]{\quantify{\forall #1, \exists #2}}

\newcommand\changes[3]{{#1} \dblcolon {#2} \leadsto {#3}}
\newcommand\D\Delta
\newcommand\init{\text{init}}
\newcommand\update{\text{update}}
\newcommand\Set{\text{Set}}
\newcommand\State{\text{State}}
\newcommand\lto\multimap

\newcommand\todo[1]{\ensuretext{\color{OrangeRed}#1}}


\begin{document}

\section{Monoids acting on themselves}

One simple way of representing values which may change is a \emph{monoid,} namely some triple $(A,+,0)$ of a set $A$, an operator ${+} : A \times A \to A$, and a zero element $0 : A$ satisfying:
%
\begin{align*}
  x + 0 &= x\\
  0 + x &= x\\
  x + (y + z) &= (x + y) + z
\end{align*}

\noindent
We ``change'' values by adding to them with $+$, while $0$ represents ``no change.''

\begin{definition}
  We write $f : A \lto B$ to indicate that $f$ is a \emph{monoid homomorphism} from $A$ to $B$, that is, that $f(0_A) = 0_B$ and $f(x +_A b) = f(x) +_B f(y)$.
\end{definition}

\begin{definition}[State monoid]
  Given a set $S$ and a monoid $(A,+,0)$, we define the monoid $\State(S,A)$ of $A$-valued $S$-transformations as:
%
  \begin{align*}
    \State(S,A) &= S \to S \times A\\
    0_{\State(S,A)} &= s \mapsto (s, 0_A)\\
    %% \text{seq}(f,g)(s) &=
    %% \text{let}~ (s_1,a_1) = g(s) ~\text{in let}~ (s_2,a2) = f(s_1) ~\text{in}~
    %% (s_2, a_1 + a_2)
    (f +_{\State(S,A)} g) &= s \mapsto (s'', a_1 +_A a_2)
    ~\text{where}~ (s', a_1) = f\<s ~\text{and}~ (s'', a_2) = g\<s'
  \end{align*}
\end{definition}

\newcommand\initout{\init^{\text{out}}}
\newcommand\initstate{\init^{\text{state}}}

\begin{definition}
  An \emph{incremental map} $f : A \to B$ between monoids $A, B$ is a tuple $(S_f, \initout_f, \initstate_f, \update_f)$ where:
  \begin{align*}
    S_f &: \Set
    &
    \initout_f &: B
    &
    \initstate_f &: S_f
    &
    \update_f &: A \lto \State({S_f}, B)
  \end{align*}
\end{definition}

\noindent
The idea is that elements of $S_f$ represent the states of our incremental operator, and $\initstate_f$ is our initial such state.
Then $\update_f(a)(s)$ takes an input delta $a : A$ and a state $s : S_f$ and produces a pair $(s', b)$ of an updated state $s' : S_f$ and an output delta $b : B$.
Finally, $\initout_f$ represents the ``initial output'' that we produce even with an input of $0_A$.
Putting this together we can recover the function we are incrementalizing:

\begin{definition}
  The \emph{underlying function} $|f| : A \to B$ of an incremental map $f : A \to B$ is given by:
%
  \begin{align*}
    |f|(a) &= \initout_f + b  ~\text{where}~(s,b) = \update_f(a)(\initstate_f)
  \end{align*}
\end{definition}

\noindent
\todo{TODO: Prove determinism of incremental maps.}
\\
\todo{TODO: Show how to compose incremental maps.}


\section{Change actions}

Change actions generalize monoids by allowing the type of \emph{deltas} to differ from the type of \emph{values}.
Deltas must form a monoid, but values may not; deltas must act on values in a way that respects the monoid structure.
%% Change actions enrich change structures by requiring that $\D A$ form a \emph{monoid} and $\oplus_A$ be a \emph{monoid action} on $|A|$.
These are \emph{monoid actions,} another well-known mathematical structure, but their application in formalizing incremental computation originates (to my knowledge) in the work of Mario Alvarez-Picallo, e.g.\ ``Fixing Incremental Computation''.

\begin{definition}
  A \emph{change action} $A$ is a tuple $(|A|, \D A, \oplus_A)$ where:
%
  \begin{align*}
    |A| &: \Set
    &
    \D A &= (|\D A|, +, 0) \ \text{is a monoid}
    &
    \oplus_A &: |A| \times \D A \to |A|
  \end{align*}
%
  such that:
%
  \begin{align*}
    a \oplus 0 &= a
    &
    a \oplus (da + da') &= (a \oplus da) \oplus da'
  \end{align*}
%
  In other words, $\oplus_A$ is a monoid action of $\D A$ on $|A|$.
\end{definition}

\begin{definition}
  A \emph{incremental map} $f : A \to B$ between change actions $A,B$ is a tuple $(S_f, \init_f, \update_f)$ where
%
  \begin{align*}
    S_f &: \Set
    \\
    \init_f &: |A| \to S_f \times |B|
    \\
    \update_f &: \D A \lto \State({S_f}, \D B)
  \end{align*}
\end{definition}

%% \begin{remark}
%%   Observe that our coherence conditions have \emph{vanished!}
%%   This is because we have managed to entirely capture them in the requirement that $\update_f$ be a monoid homomorphism from $\D A$ to $\State(S_f, \D B)$. \todo{TODO: prove this!}
%% \end{remark}


\section{Change structures}

Change structures represent a type equipped with a type of deltas, \emph{without} assuming that deltas form a monoid.
Their definition is given by Paolo Giarrusso in his PhD thesis, although I believe he calls the definition I give here ``basic change structures''.
Their roots lie in an earlier paper ``A theory of changes for higher-order languages'' by Cai et al.

\begin{definition}
  A \emph{change structure} $A$ is a triple $(|A|, \Delta A, V_A)$ of two sets $|A|, \Delta A$ and a ternary relation $V_A \subseteq \Delta A \times |A| \times |A|$.
  We will write $\changes{dx}{x}{y} : A$ to denote $(dx, x, y) \in V_A$ and gloss this as ``$dx$ changes $x$ into $y$''.
  In almost all useful change structures, this relation is functional, meaning that if $\changes{dx}{x}{y}$ and $\changes{dx}{x}{y'}$ then $y = y'$.
\end{definition}

\begin{definition}
  For change structures $A,B$, a \emph{derivative} of a map $f : |A| \to |B|$ is a map $f' : |A| \times \D A \to \D B$ such that
%
  \[
  \fa{dx,x,y} \changes{dx}{x}{y} : A \implies \changes{f'\<(x,dx)}{f\<x}{f\<y}
  \]
%
  In other words, it converts changes to f's input into changes to f's output.
\end{definition}

\begin{definition}
  A \emph{incremental map} $f : A \to B$ between change structures $A,B$ consists of a triple $(S_f, \init_f, \update_f)$ with these types:
%
  \begin{align*}
    S_f &: \Set
    &&\text{the state type}
    \\
    \init_f &: |A| \to S_f \times |B|
    &&\text{the initializer map}
    \\
    \update_f &: S_f \times \D A \to S_f \times \D B
    &&\text{the update map}
  \end{align*}
%
  satisfying the \emph{coherence conditions} given by \cref{def:coherent-incremental-map}.
\end{definition}

\begin{definition}
  The \emph{underlying function} $|f| : |A| \to |B|$ of an incremental map $f : A \to B$ is defined by $|f|(a) = b ~\text{whenever}~ \init_f(a) = (s,b)$.
\end{definition}

\begin{definition}
  The \emph{reachability relation} $R_f \subseteq S_f \times |A|$ of an incremental map $f : A \to B$ is defined inductively by:
%
  %% \begin{mathpar}
  %%   \infer{f(a) = (s,b)}{R(a,b,s)}
  %%
  %%   \infer{R(a,b,s) \\ \changes{da}{a}{a'} \\ f_\D(s,da) = (s',db) }
  %%   {\ex{b'} \changes{db}{b}{b'} \\ R(a', b', s') }
  %% \end{mathpar}
  \begin{mathpar}
    \infer{f(a) = (s,b)}{R_f(s,a)}

    \infer{R_f(s,a) \\ \changes{da}{a}{a'} \\ \delta_f(s,da) = (s',db) }
    {R_f(s', a') }
  \end{mathpar}
\end{definition}

\begin{definition}\label{def:coherent-incremental-map}
  An incremental map is \emph{coherent} if
  $R_f(s,a)$ and $\changes{da}{a}{a'} : A$ and $\update_f(s, da) = (s', db)$ imply $\changes{db}{|f|(a)}{|f|(a')} : B$.
  Equivalently \todo{(why?)}, if there exists a relation $T \subseteq S_f \times |A| \times |B|$ which makes the following inference rules hold:
%
  \begin{mathpar}
    \infer{f(a) = (s,b)}{T(s,a,b)}

    \infer{T(s,a,b) \\ \changes{da}{a}{a'} \\ \update_f(s,da) = (s',db)}
    {\ex{b'} \changes{db}{b}{b'} \\ T(s',a',b')}
  \end{mathpar}

  \noindent
  \todo{TODO: Check these conditions against Definition 20 in Mamouras' \href{https://kmamouras.github.io/papers/streams-ESOP'20.pdf}{\emph{Semantic Foundations for Deterministic Dataflow and Stream Processing}} -- they should be equivalent?}
\end{definition}

\noindent
I am indebted to Runqing Xu's CoqPL 2025 presentation ``A Framework for Differential Operators'' for providing the insight behind \cref{def:coherent-incremental-map}, although it is not quite identical to any definition given there (so far as I can tell).

\begin{definition}
  An \emph{update function} for a change structure $A$ is some $\oplus_A : |A| \times \D A \to |A|$ such that $\fa{a,da} \changes{da}{a}{a \oplus_A da}$.
\end{definition}

\begin{remark}
  Observe that, from a function $f : |A| \to |B|$, a derivative $f' : |A| \times \D A \to \D B$, and an update function $\oplus_A : |A| \times \D A \to |A|$, we can construct an incremental map $g : A \to B$ as follows:
  \begin{align*}
    S_g &= |A|
    &
    \init_g(a) &= (a, f(a))
    &
    \delta_g(a,da) &= (a \oplus da, f'(a,da))
  \end{align*}
\end{remark}

\noindent
\todo{Can we go in reverse: construct a derivative from an incremental map? Probably not.}


\end{document}
